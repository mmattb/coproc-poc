%%%%%%%%%%%%%%%%%%%%%%%%%%%%%%%%%%%%%%%%%%%%%%%%%%%%%%%%%%%%%%%%%%%%%%%%
%    INSTITUTE OF PHYSICS PUBLISHING                                   %
%                                                                      %
%   `Preparing an article for publication in an Institute of Physics   %
%    Publishing journal using LaTeX'                                   %
%                                                                      %
%    LaTeX source code `ioplau2e.tex' used to generate `author         %
%    guidelines', the documentation explaining and demonstrating use   %
%    of the Institute of Physics Publishing LaTeX preprint files       %
%    `iopart.cls, iopart12.clo and iopart10.clo'.                      %
%                                                                      %
%    `ioplau2e.tex' itself uses LaTeX with `iopart.cls'                %
%                                                                      %
%%%%%%%%%%%%%%%%%%%%%%%%%%%%%%%%%%
%
%
% First we have a character check
%
% ! exclamation mark    " double quote  
% # hash                ` opening quote (grave)
% & ampersand           ' closing quote (acute)
% $ dollar              % percent       
% ( open parenthesis    ) close paren.  
% - hyphen              = equals sign
% | vertical bar        ~ tilde         
% @ at sign             _ underscore
% { open curly brace    } close curly   
% [ open square         ] close square bracket
% + plus sign           ; semi-colon    
% * asterisk            : colon
% < open angle bracket  > close angle   
% , comma               . full stop
% ? question mark       / forward slash 
% \ backslash           ^ circumflex
%
% ABCDEFGHIJKLMNOPQRSTUVWXYZ 
% abcdefghijklmnopqrstuvwxyz 
% 1234567890
%
%%%%%%%%%%%%%%%%%%%%%%%%%%%%%%%%%%%%%%%%%%%%%%%%%%%%%%%%%%%%%%%%%%%
%
\documentclass[12pt]{iopart}
\newcommand{\gguide}{{\it Preparing graphics for IOP Publishing journals}}
%Uncomment next line if AMS fonts required
%\usepackage{iopams}
\begin{document}

\title[Modeling a Neural Co-Processor]
{Towards a Neural Co-Processor Which Restores Movement After Stroke: Modeling a Proof-of-Concept}

\author{Matthew J Bryan$^{1}$, Linxing Preston Jiang$^{1}$, Rajesh P N Rao$^{1}$}

\address{$^{1}$ Neural Systems Laboratory, Department of Computer
Science and Engineering, University of Washington, Box 352350,
Seattle, WA 98105, USA}

\ead{matthew.bryan@dell.com}
\vspace{10pt}
\begin{indented}
\item[]September 2021
\end{indented}

\begin{abstract}
Brain co-processors\cite{rao.coproc} are devices which use artificial intelligence (AI) and closed-loop
neurostimulation to shape neural activity and to bridge injured neural circuits for
targeted repair and rehabilitation.
\end{abstract}

\vspace{2pc}
\noindent{\it Keywords}: brain-computer interface, neural co-processor, ai, machine learning, stimulation
%
% Uncomment for Submitted to journal title message
%\submitto{\JPA}
%
% Uncomment if a separate title page is required
\maketitle
% 
% For two-column output uncomment the next line and choose [10pt] rather than [12pt] in the \documentclass declaration
%\ioptwocol
%



\section{Introduction}


Closed-loop control:
\begin{itemize}
	\item \cite{walker.inception} Inception Loops: driving brain states
	\item \cite{kahana.biomarker} “Closed loop” here refers to when to apply stimulation, always of the same type and at the same site, based on memory performance.
	\item \cite{tafazoli.acls} ACLS
\end{itemize}


Simulation:
\begin{itemize}
	\item \cite{bernal.sim} Simulation of spiking neural network, learning stimulation regime
\end{itemize}

\section{Method}
asdf

\section{Results}
asdf

\section{Discussion}
asdf

\section{Conclusion}
asdf

\section{Acknowledgements}
asdf

\section{Ethical Statement}
asdf

\section{References}
\bibliographystyle{iopart-num}
\bibliography{refs}
\end{document}

